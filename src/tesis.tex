\documentclass[a4paper, twopager]{article}
\author{Pablo Santoro}
\usepackage[spanish]{babel}
\usepackage[utf8]{inputenc}
\title{Trabajo de especialización - Maestría en minería de Datos - Universidad de Buenos Aires}
\begin{document}
\maketitle
\newpage
\tableofcontents
\newpage
\section{Introducción}
Este trabajo es la continuacion de la presentación del trabajo presentado para el VAST Challenge 2018 (Mini Challenge 2) por parte del equipo de Juan Pablo Pilorget, Rubén Flecha y Pablo Santoro de la Maestría en Minería de Datos de la Universidad de Buenos Aires, como trabajo final de la materia de Visualización de la Información. En el siguiente trabajo contaré en qué consistió la problemática a resolver como parte del VAST Challenge, la visualización propuesta como parte de la solución a la problemática hallada en el trabajo presentado, la mejora a la solución presentada y los desafíos a futuro para enriquecer la herramienta.
\section{Problemática}
% Contar el problema según es planteado por el VAST challenge
\section{Métodos}
% Uso de R como base para el trabajo
\section{Problemas en el muestreo}
% TODO: mostrar los graficos de deficiencias en el muestreo de la siguiente manera:
% 	1: frecuencia de muestreo irregular, da lugar a series irregulares incluso agregando por mes. Probar agregaciones bi, tri, cuatri y semestrales y anuales para ver si la escala no influye tanto.
% 	2: cantidades de muestreo muy grandes por dia, mas de una por día: cuando un 0.
\section{Conclusión}
\section{Desafíos a futuro}
Como hemos visto en el trabajo, la información presentada es compleja en cuanto a la cantidad de mediciones realizadas, el hecho de que sean todas series de tiempo, que hayan sido tomadas a intervalos irregulares de tiempo distintas entre sí y la correlación espacial (cuencas fluviales y distancia al lugar donde se arrojan los residuos). Por ende, la continuación de la aplicación presentada en este trabajo debe incorporar las complejidades del análisis de series de tiempo y espaciales de forma tal que sea inmediato al usuario las complejidades que revisten estos datos, y las complejidades que puedan tenerse a futuros.
\end{document}

